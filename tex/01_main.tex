\documentclass[11pt, class=article, crop=false]{standalone}
\usepackage[subpreambles=true]{standalone}
\usepackage[T1]{fontenc} % for font setting
\usepackage{newtxtext,newtxmath}
\usepackage{import,
            graphicx,
            parskip,
            url,
            amsmath,
            wrapfig,
            fancyhdr,
            soul,
            tabularx,
            authblk,
            textcomp,
            lineno}

% side caption figure
\usepackage{sidecap}
\sidecaptionvpos{figure}{t}

% for special characters in bibliography            
\usepackage[utf8]{inputenc}
\usepackage[T1]{fontenc}

% citation setup
\usepackage[euler]{textgreek}
\usepackage[sort\&compress]{natbib}
\setcitestyle{square}
\setcitestyle{comma}
\bibliographystyle{pnas-new}

% caption setup
\usepackage[font = small, labelfont = {bf, small}]{caption}
           
% margin
\usepackage[top=2.54cm, bottom=2.54cm, left=2.54cm, right=2.54cm]{geometry}

% title
\title{Body size and local density explain movement patterns in stream fish}
\date{} % remove date from title

% author list
\author[1]{Ashley LaRoque}
\author[2]{Seoghyun Kim}
\author[1]{Akira Terui}
\affil[1]{Depatment of Biology, University of North Carolina at Greensboro}
\affil[2]{XXX}

\linenumbers


\begin{document}

\maketitle

\section{Abstract}

\section{Introduction}

Local communities are dynamically linked to each other via movement with important consequences for the spatial organization of species distributions. Although prevailing spatial paradigms treat this ecological process as stochastic, a parallel line of research suggests that movement is a process of individual decision-making. Individuals perceive a variation in biotic and abiotic conditions among localities and may respond differently due to different movement capabilities \citep{tesfaygebrekirosFactorsAffectingStream2016}. Movement does not come without a cost, however. Mobile individuals must make an appropriate decision on when they move to maintain their fitness, such as growth and survival. Thus, movement is influenced by both extrinsic and intrinsic factors and acts as a behavioral mechanism to structure communities in space \citep{leiboldMetacommunityConceptFramework2004, mcpeekEvolutionPassiveDispersal2024, schlagelMovementmediatedCommunityAssembly2020}. 

Population density – either intraspecific or interspecific – exemplifies an extrinsic driver, motivating movement through competitive and mutualistic interactions. For example, competitive interactions may cause individuals to move away from densely populated areas, while mutualistic interactions may facilitate intra- or interspecific aggregation \citep{thierryInterplayAbioticBiotic2024, rasmussenIndividualMovementStream2017}. In parallel, the capacity to move may be affected by intrinsic individual conditions, such as body size \citep{clobertDispersalEcologyEvolution2012}. For example,  larger individuals tend to move long distances due in part to their greater locomotive capabilities, although the nature of correlation varies greatly among species and ecological contexts \citep{comteEvidenceDispersalSyndromes, teruiParasiteInfectionInduces2017, radingerPatternsPredictorsFish2014, }  (Debeffe et al., 2012; Gilliam \& Fraser, 2001). Hence, extrinsic and intrinsic drivers are critical for understanding community dynamics.

Despite this recognition, the interplay between extrinsic and intrinsic factors on movement has rarely been considered. Most previous studies have evaluated these factors in isolation due to field constraints and statistical complexity. However, this makes it difficult to understand how movement patterns may persist in nature because these drivers lack mutual exclusivity \citep{mcmahonLinkingHabitatSelection2006}. Stream fishes serve as an excellent model system to close this gap. Their movement is restricted to a one-dimensional system, making the direct observation of movement processes highly tractable in the wild. In addition, as fish individuals of various sizes move across habitat patches, they engage in inter- and intraspecific interactions over space and time \citep{brownHabitatHeterogeneityActivity2010, davidsonSeasonalSpatialHydrological2012, robinsonEffectsMultiyearExperimental2003, albaneseEcologicalCorrelatesFish2004, nakayamaFinescaleMovementEcology2018, pettyRestrictedMovementMottled2004, robertsSpatiotemporalVariabilityStream2007}. These dynamic processes can occur over small spatial scales \citep{teruiNonrandomDispersalSympatric2021}. 

Here, we used mark-recapture data to evaluate movement patterns of fish in response to the extrinsic drivers: population density; and the intrinsic driver: body size. Many stream fish movement studies only focus on explaining how one driver is responsible for movement patterns, thus narrowing the applicability of the results. This widely overlooks the complexity of movement and may bias results. We tested the following predictions: (1) larger individuals move more often than their smaller counterparts; (2) higher conspecific density areas will drive more movement among conspecifics; and (3) higher heterospecific densities will drive more movement among the competitively inferior species.

\section{Method}

\subsection{Study Site and Species}

Our study was conducted in a second-order stream stemming from the Reedy Fork River, located in the Piedmont region of North Carolina, USA (36.169939ºN, 79.722088ºW). This stream consists of riffle-pool sequences with its substrate ranging from silt to bedrock and depths ranging from a few millimeters to a meter deep at base-flow conditions. 

We selected a 430-m reach of the stream for our mark-recapture research. Four species dominated in the study reach: two cyprinids (bluehead chub Nocomis leptocephalus and creek chub Semotilus atromaculatus) and two centrarchids (green sunfish \textit{Lepomis cyanellus} and redbreast sunfish \textit{Lepomis auritus}). All four species were found throughout the entire reach and are known to exhibit resident (non-migratory) life-history strategies (CITE). Other species to persist (ordered from common to rare) within this reach include: tessellated darter Etheostoma olmstedi, striped jumprock Moxostoma rupiscartes, bluegill Lepomis macrochirus, largemouth bass Micropterus salmoides, margined madtom Noturus insignis, speckled killifish Fundulus rathbuni, notchlip redhorse Moxostoma collapsum, white shiner Luxilus albeolus, creek chubsucker Erimyzon oblongus, eastern mosquitofish Gambusia holbrooki, pond loach Misgurnus anguillicaudatus, warmouth Lepomis gulosus, rosy side dace Clinostomus funduloides, and yellow bullhead Ameiurus natalis. 

\subsection{Fish Sampling}

We focused on the dominant fish species in the study reach for our mark-recapture study. Mark-recapture sampling took place regularly from November 2020 to August 2024 at an interval of roughly three months (average 93 days), except for one interval (November 2021 to May 2022; 171 days) in which we were unable to conduct a field survey in February 2022 for logistical reasons. As a result, we collected mark-recapture data for 15 occasions.

The study reach was divided into 10-m sections to locate fishes at a fine spatial scale. In each section, fish were collected via single-pass electrofishing (Smith-Root, Inc.) and placed in a five-gallon bucket for handling. All captured individuals were identified to species and measured for total length (mm). Individuals of the study species (> 60 mm in total length) were anesthetized with MS-222 (Tricaine-S) and implanted with a 12-mm passive integrated transponder (PIT) tag to uniquely identify each individual in subsequent recaptures (Oregon RFID). This technique has been described in more detail by Cary et al. (2017). These individuals were also weighed (g). A liquid skin glue was used on the incision site for individuals of the Cyprinidae family to prevent tag loss due to the nature of their scales. After the successful implantation, fish were returned to their holding bucket and monitored to ensure survival before being released back into their section of capture. 

\subsection{Environmental Variables}

Habitat variables were measured at baseflow conditions along three evenly spaced transects per section. In each transect, we measured the following physical variables at the center and near both sides of the bank, totaling nine measurements per section: water depth (nearest cm), current velocity, and dominant substrate type (silt = <0.01mm, sand = 0.1-2mm, gravel = 2-16mm, pebble = 16-64mm, cobble = 64-256mm, boulder = 256-512mm, and bedrock = >512mm). In addition, we measured the total aerial coverage of undercut banks and woody debris per section as these structures may represent important microhabitats for the study species. We approximated the areal coverage as the area of the rectangular, calculated as the length times the mean width measured at the three points of the structure.

We calculated the surface area of each section as the mean wetted width (measured at each transect) times the section length. Water temperature and pressure were monitored hourly using a water level logger (HOBO® Onset, Model U20L-02) deployed at the upstream end of the study reach.

\subsection{Statistical Analysis}

We used the dispersal-observation model to evaluate the effects of both intrinsic and extrinsic variables on movement behaviors. This modeling framework integrates movement and observation processes, thus accounting for imperfect detection, survival, and emigration from the study section when assessing the ecological influences on movement distances.

Movement process - we define movement as fine-scale shifts over habitat patches that occur between consecutive recapture events. Let $X_{1,i}$ and $X_{0,i}$ denote locations of recapture and capture for individual i, which were measured as the distance from the midpoint of the section to the downstream end of the study stretch. We assumed $X_{1,i}$ as a random draw from a normal distribution as:    

\begin{equation}
    X_{1, i}|X_{0, i}, \sigma_i \sim \text{Normal}(X_{0, i}, \sigma_i^2)
    \label{eq:normal}
\end{equation}

where $\sigma_i$ is the standard deviation describing the absolute distance moved between consecutive capture and recapture occasions ($|X_{1,i} - X_{0,i}|$).
We linked the standard deviation of  movement distance to predictors using a log-link function: 

\begin{equation}
    \ln \sigma_i = \beta_0 + \sum_{k} \beta_k x_{k,i} + \ln \eta_i
    \label{eq:linear-pred}
\end{equation}

where $\beta_0$ is the intercept and $\beta_k$ is the regression coefficient for the $k$-th predictor $x_{k,i}$. The log-transformed time interval between capture and recapture i [ln day] was included as an offset term to standardize the movement duration between observations. In our model, we included body size as a proxy variable for the individual’s condition, whereas the densities of bluehead chub, creek chub, green sunfish, and redbreast sunfish were included to evaluate intra- and interspecific density-dependence in movement. These densities were corrected for imperfect detection of the study species (Supporting Information). We included mean water temperature and undercut bank area to control possible effects of movement seasonality and microhabitat structure. Undercut bank (UCB) areas often provide habitat refugia harboring a considerable number of individuals. Temperature averaged for each sampling interval and incorporated into the model to accommodate seasonal variation in movement. Other habitat variables, such as depth and substrate, were not included due to their correlative nature with velocity making it difficult to discern ecological interpretation from. 

Observation process - Our capture-recapture data is the imperfect representation of movement processes because fish are recaptured only when the following conditions are satisfied simultaneously: alive, stay, and detected in the study section. Our observation model accounts for this process by describing the recapture state $Y_i$ (1 if recaptured 0 otherwise) as random draws from a Bernoulli distribution:

\begin{equation}
    Y_i \sim \text{Bernoulli}(\phi_i z_i)
\end{equation}

where $\phi_i$ is the product of survival and detection probabilities (hereafter, ``recapture'' probability) and $z_i$ is the binary latent variable indicating whether individual $i$ stayed in the study section at the recapture occasion.
The latent variable $z_i$ was determined by the observed (if recaptured) or predicted location (if not recaptured) of individual $i$ as: 

\begin{equation}
    z_i =
    \begin{cases}
        1~\text{if}~0 \le X_{1,i} \le L~\text{(stay)},\\
        0~\text{otherwise (emigrate)}.
    \end{cases}
\end{equation}

$L$ is the upstream terminal of the study reach ($L = 430$). When $X_{1,i}$ was unobserved (i.e., not recaptured), a predicted value was drawn from Equation \ref{eq:normal} through the Markov Chain Monte Carlo simulations (see below), thus accounting for the observation process when estimating the movement parameter $\sigma_i$.
This coupling of observation and movement processes accounts for permanent emigration and yields less biased estimates of movement parameters.

The model was fitted to the data for each species separately using JAGS (CITE). Markov chain Monte Carlo (MCMC) simulations were run for 20,000 iterations with a 1,000 burn-in period and we retained 1,000 samples per chain by thinning every 40 steps to calculate posterior probabilities. Model convergence was checked by ensuring that the potential scale reduction factor, referred to as R-hat, was less than 1.1 for all parameters. All statistical analyses were conducted in R version 4.4.0 (CITE).

\section{Results}

Across 15 occasions, we tagged 2,430 unique individuals of our target species with 324 of those consecutively recaptured. Movement distances ranged from 0m to 390m across species and seasons. Mean length of redbreast sunfish (98.8 $\pm$ 26.3 mm) was the largest followed by creek chubs (96.1 $\pm$ 19.3 mm), bluehead chub (92.6 $\pm$ 19.8 mm), and lastly green sunfish (79.2 $\pm$ 13.1 mm). Mean density was highest among creek chubs (0.60 $\pm$ 0.54) followed by green sunfish (0.51 $\pm$ 0.51), bluehead chub (0.30 $\pm$ 0.32), and redbreast sunfish (0.29 $\pm$ 0.50). Thus, our data captured sufficient variation in body size and population densities. 

Our model revealed that movement based on length and density is species-specific with similar trends exhibited at the family-level (Figure 1). In Cyprinid species, both species tended to increase movement distance with their body size, although the effect was insignificant for bluehead chub (Figure 2). Body size had contrasting influences on Centrachid species, with larger individuals moving more in green sunfish (Figure 2). However, the effect of size on redbreast sunfish was insignificant.

Cyprinid species (creek chub and bluehead chub) responded vaguely to population densities, except for the negative effect of redbreast sunfish on creek chub (Figure 3). Centrarchid species responded more clearly to population densities. Green sunfish responded to high densities of redbreast sunfish by reducing movements. Both species increased their movements as the density of creek chub increases, while decreasing movements in areas with a high density of bluehead chub (Figure 3). No species responded significantly to their conspecific densities.

Both Centrarchid species responded to undercut bank areas. Green sunfish movements increased as UCB areas increased, while redbreast sunfish movements declined. Neither Cyprinid species was impacted by the UCB area. Temperature was only significant for bluehead chub by minimizing movements as temperature increased. 

\section{Discussion}

Movement is shaped by species responses to extrinsic and intrinsic drivers \citep{clobertDispersalEcologyEvolution2012}. Yet, these factors are often studied in isolation, causing a lack of quantitative analysis comparing their relative influences. Our four-year data of mark-recapture research unveiled that both extrinsic (population densities) and intrinsic (body size) factors influence stream fish movements with similar trends exhibited among families. Centrarchids (green sunfish and redbreast sunfish) react more clearly to the extrinsic variable, while weak patterns persist among cyprinids. Conversely, cyprinids behave akin to one another in response to the intrinsic variable, whereas centrarchids show contrasting responses. While these movement patterns seem complicated, understanding species-level movement responses can provide greater insight into how interacting species are distributed through non-random movement. 

Centrarchid species (green sunfish and redbreast sunfish) were responsive to population densities. In particular, both species moved longer distances as the density of creek chub increased. This positive interspecific density-dependence may be indicative of competition between species eluding that the sunfish are inferior competitors to the creek chub \citep{jacobHabitatChoiceMeets2018}. The diets of both cyprinids and centrarchids overlap significantly in small streams (collar et al 2009, lemly 1985, Leonard & Orth 1986, Karr 1981), potentially explaining the mechanism behind this interpretation. Creek chubs are likely superior competitors for food resources with warm temperatures \cite{tangiguchiTemperatureMediationCompetitive1998}. Centrarchid species, however, responded contrastingly to high bluehead chub densities by exhibiting less movement. Predator-prey interactions may underlie this negative interspecific density-dependence \citep{jacobHabitatMatchingSpatial2015}. Redbreast sunfish have been found predating on individuals of the Cyprinidae family \citep{borrelliPuttingLakeTogether2023}, so perhaps the bluehead chub acts as a potential prey species discouraging sunfish to move because of accessible food resources. This predation of bluehead chub has been documented to occur in green sunfish \citep{lemlySuppressionNativeFish1985}, making it probable in redbreast sunfish as well. Alternatively, shared environmental cues between the species, such as greater microhabitat (undercut bank, wood debris) availability, could explain a decrease in movement because these areas provide added refugia \citep{careyEffectsLittoralHabitat2010}. While we cannot rule out the possibility of shared environmental cues, this mechanism is unlikely because the effects of microhabitat availability were statistically controlled in our model. Although experimental approaches are needed to elucidate biological causality, the trophic mechanism is more likely in our study system. 

Cyprinid species (bluehead chub and creek chub) showed unclear responses to population density. Redbreast sunfish density was the only significant relationship showing that higher densities inhibit creek chub movement. Similar to the story above, predator-prey interactions may explain this observed pattern. Creek chubs are known omnivores feeding on a variety of insects, detritus, and fish \citep{}(champagne , leonard 1986, and Quist, Bower, and Hubert 2006). While there is direct evidence for redbreast predating upon creek chubs, it is conceivable that larger creek chubs predated on smaller redbreast sunfish given the omnivorous nature of Creek Chubs.

Interestingly, regardless of fish families, none of the study species responded significantly to intraspecific densities. This result is somewhat surprising given the wealth of studies showing intraspecific competition exceeding interspecific competition in fishes \citep{}(fish citations) and other taxa \citep{adlerCompetitionCoexistencePlant2018, barabasEffectIntraInterspecific2016, thompsonProcessbasedMetacommunityFramework2020}(other taxa). While high conspecific density could impose intensive intraspecific competition, it also increases the likelihood of finding potential mates (e.g. Allee effects)\citep{terui 2015}. In our system, this may overshadow the relative competition for other resources like habitat and food, thus producing stronger interspecific competition. This effect can be exacerbated when body size is considered because larger individuals can impose stronger interspecific competition on smaller individuals that utilize the same resource. Although further studies are needed to illuminate underlying mechanisms, these opposing influences of conspecific density could potentially obscure the influence on movement patterns. 

Movement requires great energetic expenditure that needs to be balanced among an individual’s needs ((Boisclair \& Leggett, 1989; Jobling, 1995) from cooke et al 2022). Body size is one such limiting factor in determining metabolism (Beamish, 1978 from Rubio-Garcia, 2020). Because larger individuals have lower relative metabolic demands than smaller individuals due to greater lipid reserves ((Brown \& Braithwaite, 2004; Krause et al., 1998) from kanno et al 2023), they may be better able to balance the cost of moving (Schlaegel et al 2020). Not only are larger individuals at an energetic advantage, but they may serve as a stronger competitor in the settlement phase (Rasmussen and Belk, 2015). This concept, in conjunction with our results that high densities of creek chubs drive away sunfish, may further support the idea that creek chubs are a superior competitor in the community. While this may also explain, in part, the movement pattern of larger bodied green sunfish, it is more ambiguous and may be slightly biased due to the frequency of movement among recaptures used in our statistical model. 

Movement is complex and nature cannot always follow a clear pattern as exemplified by our results. While we selected an extrinsic and intrinsic variable representative of species interactions and individual movement ability, other potential variables may be playing a role in driving movement (i.e. sex, disturbance, etc) that we were either unable to quantify or were correlated making it difficult to tease apart any ecological interpretation. Further, differences in movement patterns may arise when considering a study’s temporal frequency and duration or when utilizing an experimental vs an in situ stream. Additionally, we experienced relatively low consecutive recapture rates, likely because of the small spatial extent of our study area. Our Bayesian model accounted for emigration, imperfect detection, and survival processes to minimize statistical biases arising from this limitation. Yet, it is not possible to completely eliminate the limitation of our mark-recapture approach. Future studies addressing these limitations could provide deeper insight into the ecological divers of movement.

Movement mediates how species interact shaping communities and metacommunities \citep{schlagelMovementmediatedCommunityAssembly2020
}. Our study demonstrates how intrinsic (e.g. body size) and extrinsic (e.g. population density) drivers can help pinpoint potential movement patterns in freshwater fishes with variation among species elucidating a potential avenue to explore community dynamics in the future. Extrapolating how non-random movements, such as those identified in this study, affect species coexistence and the spatial patterns of community organization will be essential in connecting behavioral ecology to spatial theory. Metapopulation and metacommunity, two frameworks included in spatial theory, assume movement or dispersal processes are stochastic. However, this downplays the ecological mechanisms that motivate movement. To overcome this disparity, recent theoretical studies have begun to incorporate non-random movement into their simulation frameworks, yet empirical validations lag far behind. Hence, integrating field-based movement patterns with theoretical simulations will provide greater insight into the mechanisms behind community and metacommunity dynamics.

\bibliography{tex/zotero_references}

\end{document}